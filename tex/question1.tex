A build pipeline is a linked set of stages that represent the process of releasing a new feature (or bug fix etc.) into production.
The key stages are usually, based on Figure 5.4 of Humble and Farley's book~\cite{DeployPipe}:
\begin{itemize}
  \item Fetching dependencies (often fixing library versions e.g. JavaScript's \texttt{yarn.lock} or Haskell's \texttt{stack})
  \item Building the patch against the master branch/trunk
  \item Running automated test suites
  \item Bundling artefacts (e.g. Docker image or \texttt{.ipa} file for iOS app store)
  \item Deploying to a staging/production environment \textit{(optionally automated)}
\end{itemize}

To encourage rapid iteration, the cost (both in time and money) of building and deploying the \textit{product} needs to be low.
Automated build pipelines are beneficial because these stages can be run automatically as soon as the preceding step has completed,
manually or at a set time of the day.
It completely depends on what your company/team wants to do, affording you great flexibility in your process.
Within each stage you can have multiple steps that can run in parallel,
for example unit tests, feature tests and integration tests can all run at the same time.
This can dramatically reduce the time of your builds.
You also have some guarantees that each build is reliable and reproducible,
meaning you spend less time every release cycle fixing issues with broken builds.
This is more likely to encourage more frequent releases (or more frequent merges, if the releases are not instant).
\par
There are numerous tools and services that provide methods for doing all of this for you.
You build pipeline could be an all in one solution such as Gitlab CI\footnote{\label{gitlab-ci}https://about.gitlab.com/features/gitlab-ci-cd/} or Jenkins\footnote{\label{jenkins}https://jenkins.io},
or one that stretches across many services (perhaps using Github\footnote{\label{github}https://github.com},
Team City\footnote{\label{team-city}https://www.jetbrains.com/teamcity/} for building and Octopus\footnote{\label{octopus}https://octopus.com} for deployment).
All of these tools have different methods of specifying what happens during the build.
Some systems use an Infrastructure as Code style technique (Jenkins\footnoteref{jenkins}),
others use graphical layouts (Team City\footnoteref{team-city} and Gitlab CI\footnoteref{gitlab-ci}),
simpler systems such as Travis\footnote{https://travis-ci.com} and Circle CI\footnote{https://circleci.com} use simple config files.
\par
One notable drawback of using a CI tool such as Travis, which builds your product in a isolated container, is the lack of persistent environment. This means fetching dependencies, which may take some time to download and install, has to be performed on every build. We encountered and addressed this by building a docker image for the \texttt{pdflatex} tool, which could then be easily cached by Travis (on AWS), reducing our build time down from 6 minutes to under 2 minutes.
Additionally, as the \texttt{texlive-latex-base} package is downloaded and installed \textit{exactly once} when the docker image is built, our project is    immune to external changes to the package. 
In a web application context this could apply to \texttt{npm} packages or something similar,
where there are many packages that can take some time to download and install.
\par
Ideally after the build process has been completed you should have everything you need to deploy the newest version of your software~\cite{BruntonBuild}.
This means that as much as possible, you include everything needed for the final host machine to run the application.
Or this can mean bundled packages ready to be uploaded to a private \texttt{npm} server.
In his blog post~\cite{BruntonBuild} Brunton-Spall comments on this and describes
how if you are extremely unlucky you can have changes between testing and deploying if your artefacts are not immutable.
\par
If the builds are successful and the tests pass you can deploy to some staging servers you may have for internal testing, demoing or focus group testing etc.
Due to the fact build pipelines offer reproducible builds you can deploy to multiple clients or demo environments with ease and in parallel,
by simply defining the configuration for the build in each case.
