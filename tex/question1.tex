The main problems that having a build pipeline addresses are automation and ease of delivery.
For anything much larger than a toy project you will likely want a script to run various tests,
perhaps one command for all suites, another for just unit tests etc. but running lots of scripts can become a pain.
Automating the fetching of dependencies, building the product, running various stages of tests, and collecting all of the artefacts that are produced can save developers a lot of time.
By saving developers a lot of time it will encourage developers to make releases more often,
perhaps with features turned off until the business is ready or another service has been updated.
Having an automated build server can also provide performance benefits.
Many services, such as Travis CI, have dedicated, specialised servers for building projects in parallel,
as well as caching specific versions of dependencies,
this means if new releases become available your build will not break.
\par
Most continuous integration tools also boast pipelines for deploying as well,
this is really useful because it can add guarantees that the new software you've just deployed is exactly the same as the one you tested.
Additionally if you have a staging server, a test server and multiple production servers,
each for a different client manually deploying to each instance could be a lengthy process.
Whereas using a tool such as Octopus, which affords you the luxury of repeatable,
reliable deployments to multiple instances with just a few changes to some configuration variables.

